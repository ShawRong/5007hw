\documentclass{article}
\usepackage{amsmath}
\usepackage{amssymb}
\usepackage{geometry}
\usepackage{algorithm}
\usepackage{algpseudocode}
\usepackage{bm} 
\newcommand{\R}{\mathbb{R}}
\geometry{a4paper, margin=1in}

\title{MSBD 5007 HW1}
\author{RONG Shuo}
\date{\today}

\begin{document}

\maketitle

\section*{Question1}
Given two vectors \(\bm{a} = [a_0, a_1, \cdots, a_{N - 1}]\) and \(\bm{b} = [b_0, b_1, \cdots, b_{N-1}]^T\), their \textit{circular convolution} is defined by
\[(\bm{a} \circledast\bm{b})_k = \sum_{j = 0}^{N - 1} a_j b_{k - j}, k = 0, 1, \cdots, N - 1,\]
where \(\bm{b}\) is extended periodically, i.e., \(b_{k - j} = b_{(k - j) + N}\) if \(-N \leq k-j \leq -1\). Let \(\bm{f}, \bm{g}, \) and \(\bm{h}\) be vectors in \(\R^N\). Prove that the circular convolution satisfies:

\noindent 
(a) \(\bm{f}\circledast  \bm{g} = \bm{g} \circledast \bm{f}\). 

\noindent 
(b) \(\bm{f}\circledast (\bm{g} \circledast \bm{h}) = (\bm{f} \circledast \bm{g}) \circledast \bm{h}\). 
\subsection*{(a)}
\begin{align*}
    (\bm{f} \circledast\bm{g})_k & \\  
    &= \sum_{j = 0}^{N - 1} f_j g_{k - j} \\
    &= \sum_{j = k + 1}^{N - 1} f_j g_{k - j} + \sum_{j = 0}^{k} f_j g_{k - j}  \\
    &= \sum_{j = k + 1}^{N - 1} f_j g_{N + k - j} + \sum_{j = 0}^{k} f_j g_{k - j}  \\
    &= \sum_{i = k + 1}^{N - 1} f_{N + k - i} g_{i} + \sum_{j = 0}^{k} f_j g_{k - j}  \\
    &= \sum_{i = k + 1}^{N - 1}  g_{i} f_{k - i} + \sum_{i = 0}^{k} f_{k - i} g_i  \\
    &= \sum_{i = k + 1}^{N - 1}  g_{i} f_{k - i} + \sum_{i = 0}^{k} g_i f_{k - i}  \\
    &= \sum_{i = 0}^{N - 1} g_{i} f_{k - i} \\
    &= (\bm{g} \circledast \bm{f})_k
\end{align*}
Therefore, we can conclude that: \(\bm{f}\circledast  \bm{g} = \bm{g} \circledast \bm{f}\). 

\subsection*{(b)}
\begin{align*}
    (\bm{f} \circledast (\bm{g} \circledast \bm{h}))_k &= (\bm{f} \circledast (\bm{h} \circledast \bm{g}))_k \\ 
    &= \sum_{j = 0}^{N-1} f_j(\bm{h} \circledast \bm{g})_{k - j} \\
    &= \sum_{j = 0}^{N-1} f_j \sum_{i = 0}^{N-1} h_i g_{k - j - i} \\
    &= \sum_{j = 0}^{N-1}  \sum_{i = 0}^{N-1} f_j h_i g_{k - j - i} \\
    &= \sum_{i = 0}^{N-1} \sum_{j = 0}^{N-1} f_j h_i g_{k - j - i} \\
    &= \sum_{i = 0}^{N-1} h_i \sum_{j = 0}^{N-1} f_j  g_{k - j - i} \\
    &= \sum_{i = 0}^{N-1} h_i(\bm{f} \circledast \bm{g})_{k - j} \\
    &= (\bm{h} \circledast (\bm{f} \circledast \bm{g}))_k \\
    &= ((\bm{f} \circledast \bm{g}) \circledast \bm{h})_k
\end{align*}
Therefore, we can conclude that: \(\bm{f}\circledast (\bm{g} \circledast \bm{h}) = (\bm{f} \circledast \bm{g}) \circledast \bm{h}\). 

\section*{Question2}
Let \(\bm{A} = \begin{bmatrix}
   2 &-1 &3 \\
   1 & 2& 1 \\
   -3 &-1 &2 
\end{bmatrix}\) be a \(3 \times 3\) matrix.

\noindent 
(a) Find the LU decomposition of the matrix \(\bm{A}\). The final result will look like this:
\begin{align*}
    \bm{A} = \begin{bmatrix}
        1 & 0 & 0 \\
        l_{21} & 0 & 0 \\
        l_{31} & l_{32} & 1 
    \end{bmatrix} \begin{bmatrix}
        u_{11} &u_{12} &u_{13}   \\
        0 &u_{22} &u_{23}   \\
        0 &0 &u_{33}   \\
    \end{bmatrix}
\end{align*}

\noindent 
(b) Use the result in (a) to solve the system:
\begin{align*}
    2x_1 - x_2 + 3x_3 = 3 \\
    x_1 + 2x_2 + x_3 = 4 \\
    -3x_1 - x_2 + 2x_3 = 5
\end{align*}

\subsection*{(a)}
\begin{align*}
    \begin{bmatrix}
        2 &-1 &3 \\
        1/2 & 5/2& -1/2 \\
        -3/2 &-5/2 &13/2 
    \end{bmatrix}_{k=1}   \\
    \begin{bmatrix}
        2 &-1 &3 \\
        1/2 & 5/2& -1/2 \\
        -3/2 &-1 &6 
    \end{bmatrix}_{k=2}  \\
\end{align*}
From this, we can get:
\begin{align*}
    A =  
    \begin{bmatrix}
        1 & 0  &0 \\
        1/2 & 1& 0 \\
        -3/2 &-1 &1 
    \end{bmatrix} 
    \begin{bmatrix}
        2 &-1 &3 \\
        0 & 5/2& -1/2 \\
        0 & 0 &6 
    \end{bmatrix} 
\end{align*}
\subsection*{(b)}
From (a), we know:
\begin{align*}
    \begin{bmatrix}
        1 & 0  &0 \\
        1/2 & 1& 0 \\
        -3/2 &-1 &1 
    \end{bmatrix} 
    \begin{bmatrix}
        2 &-1 &3 \\
        0 & 5/2& -1/2 \\
        0 & 0 &6 
    \end{bmatrix}
    \begin{bmatrix}
        x_1\\
        x_2\\
        x_3
    \end{bmatrix} = 
    \begin{bmatrix}
        3 \\
        4 \\
        5 
    \end{bmatrix}
\end{align*}
Assume \(\bm{L}\bm{U}\bm{x} = \bm{b}\) as \(\bm{L}\bm{y} = \bm{b}\), s.t.:
From (a), we know:
\begin{align*}
    \begin{bmatrix}
        1 & 0  &0 \\
        1/2 & 1& 0 \\
        -3/2 &-1 &1 
    \end{bmatrix} 
    \begin{bmatrix}
        y_1\\
        y_2\\
        y_3
    \end{bmatrix} = 
    \begin{bmatrix}
        3 \\
        4 \\
        5 
    \end{bmatrix}
\end{align*}
From these, we can solve:
\begin{align*}
    y_1 = 3 \\
    y_2 = 5/2 \\
    y_3 = 12 
\end{align*}
s.t.
\begin{align*}
    \begin{bmatrix}
        2 &-1 &3 \\
        0 & 5/2& -1/2 \\
        0 & 0 &6 
    \end{bmatrix}
    \begin{bmatrix}
        x_1\\
        x_2\\
        x_3
    \end{bmatrix} = 
    \begin{bmatrix}
    3 \\
    5/2 \\
    12 
    \end{bmatrix}
\end{align*}
From these, we can solve:
\begin{align*}
    x_1 = -4/5\\
    x_2 = 7/5\\
    x_3 = 2 \\
\end{align*}

\section*{Question3}
Let \(\bm{A} \in \R^{n \times n}\) be a tri-diagonal matrix(i.e. \(a_{ij} = 0 if |i - j| > 1\)). The pattern of nonzero entries is illustrated below:
\begin{align*}
    \begin{bmatrix}
        \times & \times & & & \\
        \times & \times & \times & & \\
        & \ddots & \ddots & \ddots & \\
        & &  \times & \times &\times \\
        & & & \times & \times \\
    \end{bmatrix}
\end{align*}

\noindent 
Develop an algorithm with complexity \(\textit{O}(n)\) to compute the LU decomposition of \(\bm{A}\), assuming all the pivots are non-zero.

\subsection*{ }

\begin{algorithm}
    \caption{LU Decomposition}
    \begin{algorithmic}[1]
        \For{k = 1 \text{ to } n}
            \State $a_{k,k+1} \gets \frac{a_{k,k+1}}{a_{k,k}}$
            \State $a_{k+1,k+1} \gets a_{k+1,k+1} - a_{k,k+1} \cdot a_{k+1,k}$
        \EndFor
    \end{algorithmic}
\end{algorithm}

According to the LU decomposition algorithm, we can get the updated matrix \(\bm{A^*}\), and we can get the matrix \(\bm{L}\) and \(\bm{U}\) as follows:
\(\bm{L} = \begin{bmatrix}
    a_{i, j}^*
\end{bmatrix}_{i > j}\) and \(\bm{U} = \begin{bmatrix}
    a_{i, j}^*
\end{bmatrix}_{i \leq j}\). Therefore, the algorithm has a complexity of \(O(n)\).

\section*{Question4}
To accelerate matrix multiplications, the \textit{Coppersmith Winograd} algorithm reduces the number of scalar multiplications by cleverly reformulating
the inner product. Assume that \textit{n} is even and define, for any vector \(x \in \R^n\),
\begin{align*}
    f(\bm{x}) = \sum_{i = 1}^{n / 2} x_{2i - 1} x_{2i}
\end{align*}.


\noindent 
(a) Prove that for all vectors \(\bm{x}, \bm{y} \in \R^n\), the inner product can be re-expressed as
\begin{align*}
    \bm{x}^T\bm{y} = \sum_{i = 1}^{n / 2}((x_{2i - 1} + y_{2i})(x_{2i} + y_{2i - 1})) - f(\bm{x}) - f(\bm{y}).
\end{align*}

\noindent 
(b) Now consider the matrix product \(\bm{C} = \bm{A}\bm{B}\), where \(\bm{A}, \bm{B} \in \R^{n \times n}\). Devise an algorithm to compute \(\bm{C}\) using
\(\frac{n^3}{2} + \textit{O}(n^2)\) scalar multiplications.

\noindent
\textit{Note:} A standard matrix multiplication requires \(n^3\) scalar multiplications. By combining this method with other techniques, one can obtain
the Coppersmith Winograd algorithm, which has an asymptotic complexity of \(O(n^{2.375})\).

\subsection*{(a)}
\begin{align*}
    &\sum_{i = 1}^{n / 2}((x_{2i - 1} + y_{2i})(x_{2i} + y_{2i - 1})) \\
    &=\sum_{i = 1}^{n / 2}x_{2i - 1}x_{2i} + y_{2i}x_{2i} + x_{2i - 1}y_{2i - 1} +  y_{2i}y_{2i - 1} \\
    &=\sum_{i = 1}^{n / 2}x_{2i - 1}x_{2i} + \sum_{i = 1}^{n / 2}y_{2i}x_{2i} + \sum_{i = 1}^{n / 2}x_{2i - 1}y_{2i - 1} +  \sum_{i = 1}^{n / 2}y_{2i}y_{2i - 1} \\
    &= \bm{x}^T\bm{y} + f(\bm{x}) + f(\bm{y})
\end{align*}
So we can conclude that:
\begin{align*}
    \bm{x}^T\bm{y} = \sum_{i = 1}^{n / 2}((x_{2i - 1} + y_{2i})(x_{2i} + y_{2i - 1})) - f(\bm{x}) - f(\bm{y}).
\end{align*}

\subsection*{(b)}
\noindent
For each row \(\bm{a}_i \in \bm{A}\), calculating \(f(\bm{a}_i)\) costs \(\frac{n}{2}\) scalar multiplications. Therefore, the total cost of calculating \(f(\bm{a}_i), \forall \bm{a}_i \in \bm{A}\) is \(n \times \frac{n}{2} = \frac{n^2}{2}\) scalar multiplications. Similarly, the total cost of calculating \(f(\bm{b}_i),  \forall \bm{b}_i \in \bm{B}\) is \(\frac{n^2}{2}\) scalar multiplications, too.

\noindent
For each multiplication between row in \(\bm{A}\) and column in \(\bm{B}\), we need to calculate \(\frac{n}{2}\) times multiplication, if the \(f(\bm{a}_i)\) and \( f(\bm{b}_i)\) is given. Therefore, the total cost of calculating \(\bm{A} \times \bm{B}\) is \(n \times n \times \frac{n}{2} + O(n^2) = \frac{n^3}{2} + O(n^2)\) scalar multiplications.

\noindent
Therefore, the algorithm to compute \(\bm{C} = \bm{A}\bm{B}\) is as follows:

\noindent
\begin{algorithm}
    \caption{Matrix Multiplication}
    \begin{algorithmic}[1]
        \State \textbf{Input:} Matrices \(\bm{A}, \bm{B} \in \R^{n \times n}\)
        \State \textbf{Output:} Matrix \(\bm{C} = \bm{A} \bm{B}\)
        \State Compute \(f(\bm{a}_i)\) for each row \(\bm{a}_i\) in \(\bm{A}\)
        \State Compute \(f(\bm{b}_i)\) for each row \(\bm{b}_i\) in \(\bm{B}\)
        \For{each row \(\bm{a}_i\) in \(\bm{A}\)}
            \For{each column \(\bm{b}_j\) in \(\bm{B}\)}
                \State Compute \(\sum_{k = 1}^{n / 2}((a_{i, 2k - 1} + b_{2k, j})(a_{i, 2k} + b_{2k - 1, j}))\)
                \State Compute \(c_{i, j} = \sum_{k = 1}^{n / 2}((a_{i, 2k - 1} + b_{2k, j})(a_{i, 2k} + b_{2k - 1, j})) - f(\bm{a}_i) - f(\bm{b}_j)\)
            \EndFor
        \EndFor
        \State \textbf{return} \(\bm{C}\)
    \end{algorithmic}
\end{algorithm}




\end{document}